\documentclass[a4paper]{article}
\usepackage[utf8]{inputenc}
\usepackage[english]{babel}
\usepackage{graphicx}
\usepackage{multicol}
\usepackage{amsmath}
\usepackage{hyperref}
\usepackage{amsthm}
\usepackage{geometry}
\geometry{a4paper} 
\usepackage{fancyhdr}
\usepackage{xcolor}
\usepackage{amssymb}
\usepackage{multicol}
\begin{document}
\author{\textbf{Elshimaa Ahmed}}
\title{\textbf{Math 501 \\
\large Lecture 1\\}}
\date {\today}
\maketitle
\noindent
\theoremstyle{definition}
\newtheorem{definition}{Definition}[section]
\section{Preposition}
\paragraph{}
  a preposition is a claim or a declerative statement which has a truth value , can be proven to be either $True$ or $False$
\paragraph{For Example: }
\begin{itemize}
    \item "Shimaa studies discrete mathimatics" is considered as $Preposition$
    \item "How was your day?" is considered as $ not-Preposition$
    \item $x$ + 1 = 2 is considered as $non$ $declerative$ $statement$ because it is truth value depends on a variable so cannot be proven to be $True$ or $False$ without knowing the value of that variable 
\end{itemize}
\section{Logical Operators}
     \begin{definition}[\textbf{Negation}]
       let $p$ a preposition the negation of $p$ , denoted by $\neg$$p$ .The truth value of the negation of $p$ $\neg$ $p$ is the opposite value of $p$
       , expressed in English as "It's not the case that. p" 
     \end{definition}
    \textbf{For Example:}
    \begin{itemize}
      \item the negation of "I have more than 5 friends" will become "I have at most 5 friends"
    \end{itemize}

    \begin{definition}[\textbf{Conjunction}]
      let $p$ and $q$ be prepositions , the conjunction of $p$ and $q$ denoted by $p$ $\wedge$ $q$ is a preposition "$p$ and $q$" that become true only if both $p$ and $q$ are both $True$  
    \end{definition}

    \begin{definition}[\textbf{Disjunction}] 
      let $p$ and $q$ be prepositions .The Disjunction of $p$ and $q$ denoted by $p$ $\vee$ $q$ is a preposition "$p$ or $q$" which is $False$ only if both of $p$ and $q$ are $False$

    \end{definition}
    \begin{definition}[\textbf{Exclusive Disjunction}]
      let $p$ and $q$ be prepositions .The esclusive or denoted by $p$ $\oplus$ $q$ is a preposition that is $True$ if exactly one of $p$ or $q$ are $True$ , and $False$ otherwise
    \end{definition}
\section{Conditional Statements}
\begin{definition}
  let $p$ and $q$ be prepositions . The Conditional Statements $p$ $\rightarrow$ $q$ , "if $p$ then $q$" is false whenever p is $False$ or q is $True$
\end{definition}
The meaning of $p$ $\rightarrow$ $q$ assert that $q$ is true whenever $p$ holds but not vise versa ,when $p$ is $False$ it does not matter what the value of $q$ for implication to be $True$ ,$p$ is called (hypothesis or antecedent or premise ) while $q$ is called conclusion or consequence .
\newline
\newline
\textbf{English Phrases to express conditional statements:}
\begin{multicols}{2}
\begin{itemize}
  \item "if $p$, then $q$"
  \item "$p$ implies $q$"
  \item "$p$ is sufficient of $q$"
  \item "$p$ only if $q$"
  \item "$q$ is necessary for $p$" 
  \item "$q$ unless $\neg$ $p$"  \textbf{important}
  \item "$p$ only if $q$"
  \item "$q$ whenever $p$ "
\end{itemize}
  
\end{multicols}
$\newline$ 
\textbf{Converse, Contrapositive, and Inverse:}
for conditional statement $p$ $\rightarrow$ $q$
\begin{itemize}
  \item $q$ $\rightarrow$ $p$ called the converse .
  \item $\neg q$ $\rightarrow$ $\neg p$ called the Contrapositive and has the same truth value as the original statement
  \item $\neg p$ $\rightarrow$ $\neg q$ called the inverse 
\end{itemize}
\begin{definition}
  Let $p$ and $q$ be prepositions , the biconditional statement $p$ $\leftrightarrow$ $q$ is a preposition "$p$ if and only if $q$" which is $True$ when $p$ and $q$ have the same truth values and $False$ otherwise 
\end{definition}
\section{Precedence of Logical Operators}
\begin{tabular}{|c|c|c|}
 \hline
 Logical Operator & Precedence \\
 \hline
 $\neg$ & 1\\
 \hline
 $\wedge$ & 2\\
 $\vee$ & 3\\
 \hline
 $\rightarrow$ & 4\\
 $\leftrightarrow $& 5\\
 \hline
\end{tabular}
\end{document}