\documentclass[a4paper]{article}
\usepackage[utf8]{inputenc}
\usepackage[english]{babel}
\usepackage{graphicx}
\usepackage{multicol}
\usepackage{amsmath}
\usepackage{hyperref}
\usepackage{amsthm}
\usepackage{geometry}
\geometry{a4paper} 
\usepackage{fancyhdr}
\usepackage{xcolor}
\begin{document}
\author{\textbf{Elshimaa Ahmed}}
\title{\textbf{Lecture 2}}
\date {\today}
\maketitle
\theoremstyle{definition}
\newtheorem{definition}{Definition}[section]
\section{Predicates}
\paragraph{}
predicates are statements involving variables such that " $x >3$ " . These statements are neither tru or false where the value of the variable is not specified.
The statement "$x$ is greater than 3" involving two parts . the variable part $x$ and the predicate(property) it self " great than 3" .its denoted by $P(x)$  once a value is assigned to $x$ the $P(x)$ become a preposition
\newline
\newline
\textbf{For example:}
\newline
let $P(x)$ denote the statement "$x > 3$" what are truth values of $P(4)$ and $P(2)$
\newline
\newline
\textbf{Solution: } we obtain $P(4)$ by setting $x = 4$ in the statement, Hence $P(4)$  is the statement "$3 > 4$" which is $True$ . $P(2)$ is the statement "$ 2 >3$" which is $False$ 
\section{Quantifiers}
\begin{definition}[\textbf{Universal Quantifier}]
    the universal Quantifier of $P(x)$ is the statement "$P(X)$ is $True$ for all values of $x$ in the domain". It is denoted by $\forall x P(x)$
\begin{definition}[\textbf{Esistential Quantifier}]
    the existential quantifier of $P(x)$ is the preposition " there exist an element $x$ such that $P(x)$ is $True$" . It is denoted by $\exists x P(x)$
\end{definition}    
\end{definition}

\begin{tabular}{|c|c|c|}
    \hline
    statement & when true & when false \\ 
    \hline
    $\forall x P(x)$ & P(x) is $True$ for all values of $x$ & there is an $x$ for which $P(x)$ is $False$\\
    $\exists x P(x)$ & there is an $x$ such that $P(x)$ is $True$ & $P(x)$ is $False$ for all $x$ \\
    \hline
\end{tabular}
\newline\newline
\textbf{English phrases with quatifiers:}
\begin{itemize}
    \item "no one is P(x)" $\longleftrightarrow$ $\forall x \neg P(x)$ or $\neg(\exists x P(x))$
    \item "not every one is P(x)"$\longleftrightarrow$ $\neg(\forall x P(x))$ or $\exists x\neg P(x)$
    \item "exactly one is P(x) " $\longleftrightarrow$ $\exists x (P(x) \wedge \forall y (P(y) \rightarrow x = y)) $ 
    \item "all Q(x) is P(x) " $\longleftrightarrow$ $ \forall x (Q(x) \rightarrow P(x))$
    \item "all Q(x) isnot P(x)" $\longleftrightarrow$ $\forall x (Q(x)\rightarrow \neg P(x))$
    \item "some Q(x) are P(x)" $\longleftrightarrow$ $\exists x (Q(x)\wedge P(x))$
    \item "some Q(x) are not P(x)" $\longleftrightarrow$ $\exists x (Q(x)\wedge \neg P(x))$
\newline\newline
\end{itemize}

\noindent
\textbf{Nested Quantifiers}
\newline
\newline
\begin{tabular}{|p{2cm}|p{7cm}|p{7cm}|}
    \hline
    statement & when true? &when false? \\
    \hline
    $\forall \forall y P(x,y)$ & P(x,y) is true for all possible pairs & there is a pair (x,y) such that P(x,y) is false\\
    $\forall y \forall x P(x,y)$ & &\\
    \hline
    $\forall x \exists y P(x,y)$ & for every x there is y (not necessary same for different values of x) such that P(x,y) is true & there is an x such that P(x,y) is false for every y \\
    \hline
    $\exists y \forall x P(x,y)$ & there is exist a specific (same) y that P(x,y) is true for all x & for every y there is at leasy one  x that P(x,y) fails\\
    \hline
    $\exists x \exists y P(x,y)$ & therer is a pair (x,y) for which P(x,y) is true & for all pairs (x,y) P(x,y) is False\\
    $\exists y \exists x P(x,y) $&&\\
    \hline 
\end{tabular}



\end{document}